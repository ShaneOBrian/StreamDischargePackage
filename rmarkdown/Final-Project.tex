% Options for packages loaded elsewhere
\PassOptionsToPackage{unicode}{hyperref}
\PassOptionsToPackage{hyphens}{url}
%
\documentclass[
]{article}
\usepackage{amsmath,amssymb}
\usepackage{iftex}
\ifPDFTeX
  \usepackage[T1]{fontenc}
  \usepackage[utf8]{inputenc}
  \usepackage{textcomp} % provide euro and other symbols
\else % if luatex or xetex
  \usepackage{unicode-math} % this also loads fontspec
  \defaultfontfeatures{Scale=MatchLowercase}
  \defaultfontfeatures[\rmfamily]{Ligatures=TeX,Scale=1}
\fi
\usepackage{lmodern}
\ifPDFTeX\else
  % xetex/luatex font selection
\fi
% Use upquote if available, for straight quotes in verbatim environments
\IfFileExists{upquote.sty}{\usepackage{upquote}}{}
\IfFileExists{microtype.sty}{% use microtype if available
  \usepackage[]{microtype}
  \UseMicrotypeSet[protrusion]{basicmath} % disable protrusion for tt fonts
}{}
\makeatletter
\@ifundefined{KOMAClassName}{% if non-KOMA class
  \IfFileExists{parskip.sty}{%
    \usepackage{parskip}
  }{% else
    \setlength{\parindent}{0pt}
    \setlength{\parskip}{6pt plus 2pt minus 1pt}}
}{% if KOMA class
  \KOMAoptions{parskip=half}}
\makeatother
\usepackage{xcolor}
\usepackage[margin=1in]{geometry}
\usepackage{color}
\usepackage{fancyvrb}
\newcommand{\VerbBar}{|}
\newcommand{\VERB}{\Verb[commandchars=\\\{\}]}
\DefineVerbatimEnvironment{Highlighting}{Verbatim}{commandchars=\\\{\}}
% Add ',fontsize=\small' for more characters per line
\usepackage{framed}
\definecolor{shadecolor}{RGB}{248,248,248}
\newenvironment{Shaded}{\begin{snugshade}}{\end{snugshade}}
\newcommand{\AlertTok}[1]{\textcolor[rgb]{0.94,0.16,0.16}{#1}}
\newcommand{\AnnotationTok}[1]{\textcolor[rgb]{0.56,0.35,0.01}{\textbf{\textit{#1}}}}
\newcommand{\AttributeTok}[1]{\textcolor[rgb]{0.13,0.29,0.53}{#1}}
\newcommand{\BaseNTok}[1]{\textcolor[rgb]{0.00,0.00,0.81}{#1}}
\newcommand{\BuiltInTok}[1]{#1}
\newcommand{\CharTok}[1]{\textcolor[rgb]{0.31,0.60,0.02}{#1}}
\newcommand{\CommentTok}[1]{\textcolor[rgb]{0.56,0.35,0.01}{\textit{#1}}}
\newcommand{\CommentVarTok}[1]{\textcolor[rgb]{0.56,0.35,0.01}{\textbf{\textit{#1}}}}
\newcommand{\ConstantTok}[1]{\textcolor[rgb]{0.56,0.35,0.01}{#1}}
\newcommand{\ControlFlowTok}[1]{\textcolor[rgb]{0.13,0.29,0.53}{\textbf{#1}}}
\newcommand{\DataTypeTok}[1]{\textcolor[rgb]{0.13,0.29,0.53}{#1}}
\newcommand{\DecValTok}[1]{\textcolor[rgb]{0.00,0.00,0.81}{#1}}
\newcommand{\DocumentationTok}[1]{\textcolor[rgb]{0.56,0.35,0.01}{\textbf{\textit{#1}}}}
\newcommand{\ErrorTok}[1]{\textcolor[rgb]{0.64,0.00,0.00}{\textbf{#1}}}
\newcommand{\ExtensionTok}[1]{#1}
\newcommand{\FloatTok}[1]{\textcolor[rgb]{0.00,0.00,0.81}{#1}}
\newcommand{\FunctionTok}[1]{\textcolor[rgb]{0.13,0.29,0.53}{\textbf{#1}}}
\newcommand{\ImportTok}[1]{#1}
\newcommand{\InformationTok}[1]{\textcolor[rgb]{0.56,0.35,0.01}{\textbf{\textit{#1}}}}
\newcommand{\KeywordTok}[1]{\textcolor[rgb]{0.13,0.29,0.53}{\textbf{#1}}}
\newcommand{\NormalTok}[1]{#1}
\newcommand{\OperatorTok}[1]{\textcolor[rgb]{0.81,0.36,0.00}{\textbf{#1}}}
\newcommand{\OtherTok}[1]{\textcolor[rgb]{0.56,0.35,0.01}{#1}}
\newcommand{\PreprocessorTok}[1]{\textcolor[rgb]{0.56,0.35,0.01}{\textit{#1}}}
\newcommand{\RegionMarkerTok}[1]{#1}
\newcommand{\SpecialCharTok}[1]{\textcolor[rgb]{0.81,0.36,0.00}{\textbf{#1}}}
\newcommand{\SpecialStringTok}[1]{\textcolor[rgb]{0.31,0.60,0.02}{#1}}
\newcommand{\StringTok}[1]{\textcolor[rgb]{0.31,0.60,0.02}{#1}}
\newcommand{\VariableTok}[1]{\textcolor[rgb]{0.00,0.00,0.00}{#1}}
\newcommand{\VerbatimStringTok}[1]{\textcolor[rgb]{0.31,0.60,0.02}{#1}}
\newcommand{\WarningTok}[1]{\textcolor[rgb]{0.56,0.35,0.01}{\textbf{\textit{#1}}}}
\usepackage{graphicx}
\makeatletter
\def\maxwidth{\ifdim\Gin@nat@width>\linewidth\linewidth\else\Gin@nat@width\fi}
\def\maxheight{\ifdim\Gin@nat@height>\textheight\textheight\else\Gin@nat@height\fi}
\makeatother
% Scale images if necessary, so that they will not overflow the page
% margins by default, and it is still possible to overwrite the defaults
% using explicit options in \includegraphics[width, height, ...]{}
\setkeys{Gin}{width=\maxwidth,height=\maxheight,keepaspectratio}
% Set default figure placement to htbp
\makeatletter
\def\fps@figure{htbp}
\makeatother
\setlength{\emergencystretch}{3em} % prevent overfull lines
\providecommand{\tightlist}{%
  \setlength{\itemsep}{0pt}\setlength{\parskip}{0pt}}
\setcounter{secnumdepth}{-\maxdimen} % remove section numbering
\ifLuaTeX
  \usepackage{selnolig}  % disable illegal ligatures
\fi
\IfFileExists{bookmark.sty}{\usepackage{bookmark}}{\usepackage{hyperref}}
\IfFileExists{xurl.sty}{\usepackage{xurl}}{} % add URL line breaks if available
\urlstyle{same}
\hypersetup{
  pdftitle={Final Project},
  pdfauthor={Matilde Saucedo \& Shane O'Brian},
  hidelinks,
  pdfcreator={LaTeX via pandoc}}

\title{Final Project}
\author{Matilde Saucedo \& Shane O'Brian}
\date{2024-03-06}

\begin{document}
\maketitle

\begin{Shaded}
\begin{Highlighting}[]
\FunctionTok{source}\NormalTok{(}\StringTok{"../R/stream\_data.R"}\NormalTok{)}
\FunctionTok{source}\NormalTok{(}\StringTok{"../R/discharge\_vol.R"}\NormalTok{)}
\NormalTok{discharge\_pmap }\OtherTok{\textless{}{-}} \FunctionTok{pmap}\NormalTok{(}\FunctionTok{list}\NormalTok{(}\AttributeTok{depth =}\NormalTok{ stream\_features}\SpecialCharTok{$}\NormalTok{depth, }\AttributeTok{velocity =}\NormalTok{ stream\_features}\SpecialCharTok{$}\NormalTok{velocity), discharge\_vol)}
\NormalTok{discharge\_pmap\_df }\OtherTok{\textless{}{-}} \FunctionTok{data.frame}\NormalTok{(}\AttributeTok{measure =}\NormalTok{ stream\_features}\SpecialCharTok{$}\NormalTok{measure, }\AttributeTok{discharge =} \FunctionTok{unlist}\NormalTok{(discharge\_pmap), }\AttributeTok{depth =}\NormalTok{ stream\_features}\SpecialCharTok{$}\NormalTok{depth, }\AttributeTok{velocity =}\NormalTok{ stream\_features}\SpecialCharTok{$}\NormalTok{velocity, }\AttributeTok{velocity\_type=}\NormalTok{stream\_features}\SpecialCharTok{$}\NormalTok{velocity\_type, }\AttributeTok{width=}\NormalTok{stream\_features}\SpecialCharTok{$}\NormalTok{width)}

\CommentTok{\#quick visualization}
\NormalTok{discharge\_pmap\_plot }\OtherTok{=} \FunctionTok{ggplot}\NormalTok{(discharge\_pmap\_df, }\FunctionTok{aes}\NormalTok{(velocity\_type,discharge, }\AttributeTok{fill=}\NormalTok{velocity\_type))}\SpecialCharTok{+}
  \FunctionTok{geom\_boxplot}\NormalTok{()}\SpecialCharTok{+}
  \FunctionTok{labs}\NormalTok{( }\AttributeTok{x=}\StringTok{"Streamflow"}\NormalTok{, }\AttributeTok{y=}\StringTok{"Discharge in cubic feet"}\NormalTok{)}\SpecialCharTok{+}
  \FunctionTok{ggtitle}\NormalTok{(}\StringTok{"Discharge Volumes for Different Stream Levels"}\NormalTok{) }\SpecialCharTok{+}
  \FunctionTok{theme}\NormalTok{(}\AttributeTok{legend.position=}\StringTok{"none"}\NormalTok{)}
\NormalTok{discharge\_pmap\_plot}
\end{Highlighting}
\end{Shaded}

\includegraphics{Final-Project_files/figure-latex/for pmap-1.pdf}

\begin{Shaded}
\begin{Highlighting}[]
\NormalTok{discharge\_loop\_df }\OtherTok{\textless{}{-}} \FunctionTok{data.frame}\NormalTok{(}\FunctionTok{matrix}\NormalTok{(}\AttributeTok{nrow =} \FunctionTok{nrow}\NormalTok{(stream\_features), }\AttributeTok{ncol =} \DecValTok{1}\NormalTok{))}

\NormalTok{discharge\_loop\_output }\OtherTok{\textless{}{-}} \FunctionTok{data.frame}\NormalTok{(}\AttributeTok{discharge =} \FunctionTok{numeric}\NormalTok{(),}
                              \AttributeTok{velocity =} \FunctionTok{numeric}\NormalTok{(),}
                              \AttributeTok{velocity\_type =} \FunctionTok{character}\NormalTok{(),}
                              \AttributeTok{depth =} \FunctionTok{numeric}\NormalTok{(),}
                              \AttributeTok{width =} \FunctionTok{numeric}\NormalTok{())}

\CommentTok{\# Run a loop to calculate discharge for each row in stream\_features}
\ControlFlowTok{for}\NormalTok{(i }\ControlFlowTok{in} \DecValTok{1}\SpecialCharTok{:}\FunctionTok{nrow}\NormalTok{(stream\_features)) \{}
  \CommentTok{\# Calculate discharge for the current row}
\NormalTok{  discharge\_loop }\OtherTok{\textless{}{-}} \FunctionTok{discharge\_vol}\NormalTok{(}\AttributeTok{depth =}\NormalTok{ stream\_features}\SpecialCharTok{$}\NormalTok{depth[i], }
                              \AttributeTok{velocity =}\NormalTok{ stream\_features}\SpecialCharTok{$}\NormalTok{velocity[i])}
  
  \CommentTok{\# Append discharge, velocity, depth, and width to associated\_data}
\NormalTok{  discharge\_loop\_output }\OtherTok{\textless{}{-}} \FunctionTok{rbind}\NormalTok{(discharge\_loop\_output, }
                           \FunctionTok{data.frame}\NormalTok{(}\AttributeTok{discharge =}\NormalTok{ discharge\_loop, }
                                      \AttributeTok{velocity =}\NormalTok{ stream\_features}\SpecialCharTok{$}\NormalTok{velocity[i],}
                                      \AttributeTok{velocity\_type =}\NormalTok{ stream\_features}\SpecialCharTok{$}\NormalTok{velocity\_type[i],}
                                      \AttributeTok{depth =}\NormalTok{ stream\_features}\SpecialCharTok{$}\NormalTok{depth[i], }
                                      \AttributeTok{width =}\NormalTok{ stream\_features}\SpecialCharTok{$}\NormalTok{width[i]}
\NormalTok{                                      ))}
\NormalTok{\}}

\CommentTok{\#quick visualization}
\NormalTok{discharge\_loop\_plot }\OtherTok{=} \FunctionTok{ggplot}\NormalTok{(discharge\_loop\_output, }\FunctionTok{aes}\NormalTok{(velocity\_type,discharge, }\AttributeTok{fill=}\NormalTok{velocity\_type))}\SpecialCharTok{+}
  \FunctionTok{geom\_boxplot}\NormalTok{()}\SpecialCharTok{+}
  \FunctionTok{labs}\NormalTok{( }\AttributeTok{x=}\StringTok{"Streamflow"}\NormalTok{, }\AttributeTok{y=}\StringTok{"Discharge in cubic feet"}\NormalTok{)}\SpecialCharTok{+}
  \FunctionTok{ggtitle}\NormalTok{(}\StringTok{"Discharge Volumes for Different Stream Levels"}\NormalTok{) }\SpecialCharTok{+}
  \FunctionTok{theme}\NormalTok{(}\AttributeTok{legend.position=}\StringTok{"none"}\NormalTok{)}
\NormalTok{discharge\_loop\_plot}
\end{Highlighting}
\end{Shaded}

\includegraphics{Final-Project_files/figure-latex/for looping-1.pdf}

\begin{Shaded}
\begin{Highlighting}[]
\CommentTok{\#Using a negative value for velocity}
\FunctionTok{library}\NormalTok{(testthat)}
\end{Highlighting}
\end{Shaded}

\begin{verbatim}
## 
## Attaching package: 'testthat'
\end{verbatim}

\begin{verbatim}
## The following object is masked from 'package:devtools':
## 
##     test_file
\end{verbatim}

\begin{verbatim}
## The following object is masked from 'package:dplyr':
## 
##     matches
\end{verbatim}

\begin{verbatim}
## The following objects are masked from 'package:readr':
## 
##     edition_get, local_edition
\end{verbatim}

\begin{verbatim}
## The following object is masked from 'package:tidyr':
## 
##     matches
\end{verbatim}

\begin{verbatim}
## The following object is masked from 'package:purrr':
## 
##     is_null
\end{verbatim}

\begin{Shaded}
\begin{Highlighting}[]
\FunctionTok{source}\NormalTok{(}\StringTok{"../Test/test\_stream\_data.R"}\NormalTok{)}

\NormalTok{test\_results }\OtherTok{\textless{}{-}} \FunctionTok{test\_that}\NormalTok{(}\StringTok{"Negative input values return an error"}\NormalTok{, \{}
  \FunctionTok{expect\_error}\NormalTok{(}\FunctionTok{discharge\_vol}\NormalTok{(}\SpecialCharTok{{-}}\DecValTok{1}\NormalTok{, }\DecValTok{3}\NormalTok{), }\StringTok{"Depth cannot be negative"}\NormalTok{)}
  \FunctionTok{expect\_error}\NormalTok{(}\FunctionTok{discharge\_vol}\NormalTok{(}\DecValTok{4}\NormalTok{, }\SpecialCharTok{{-}}\DecValTok{6}\NormalTok{), }\StringTok{"Velocity cannot be negative"}\NormalTok{)}
\NormalTok{\})}
\end{Highlighting}
\end{Shaded}

\begin{verbatim}
## Test passed
\end{verbatim}

\begin{Shaded}
\begin{Highlighting}[]
\NormalTok{test\_results}
\end{Highlighting}
\end{Shaded}

\begin{verbatim}
## [1] TRUE
\end{verbatim}

\begin{Shaded}
\begin{Highlighting}[]
\NormalTok{devtools}\SpecialCharTok{::}\FunctionTok{document}\NormalTok{()}
\end{Highlighting}
\end{Shaded}

\begin{verbatim}
## i Updating Final Project documentation
## i Loading Final Project
\end{verbatim}

\begin{verbatim}
## Warning: -- Conflicts ---------------------------------------- Final Project conflicts
## --
## x `discharge_vol` masks `Final Project::discharge_vol()`.
## i Did you accidentally source a file rather than using `load_all()`?
##   Run `rm(list = c("discharge_vol"))` to remove the conflicts.
\end{verbatim}

\begin{verbatim}
## x stream_data.R:8: Block must have a @name.
## i Either document an existing object or manually specify with @name
## x Skipping 'NAMESPACE'
## i It already exists and was not generated by roxygen2.
\end{verbatim}

\end{document}
